\documentclass[a4paper]{article}
\usepackage[utf8]{inputenc}
\usepackage[ngerman]{babel}

\renewcommand{\thesection}
{§~\arabic{section}}


\author{Armin Bauer\\
Gregor Jehle\\
Lasse Pommerenke}
\date{10. Dezember 2012}
\title{Satzung des Instituts für angewandte Futuristik}

\begin{document}
\maketitle


\section{Name und Sitz}\label{sec:name_und_sitz}

\begin{enumerate}
\item Der Verein führt den Namen "`Institut für angewandte Futuristik"', abgekürzt "`infuanfu"'.
\item Er führt nach Eintragung in das Vereinsregister den Namenszusatz "`eingetragener Verein"' in der abgekürzten Form "`e. V."'.
\item Der Verein hat seinen Sitz in Stuttgart.
\end{enumerate}


\section{Zweck des Vereins}\label{sec:zweck_des_vereins}

Zweck des Vereins ist die
\begin{enumerate}
\item Veröffentlichung von Studien, Forschungs- und anderen Arbeitsergebnissen zur gemeinsamen Vergrößerung des persönlichen Ruhms.
\item Fort- \& Weiterbildung seiner Mitglieder
\item Verifikation und Zertifizierung von gemeinsam erworbenem Wissen
\item Widerlegung von pseudowissenschaftlichen Behauptungen, der kritische Diskurs zu Esoterik, insbesondere Homöopathie, BWL und VWL, sowie die Überprüfung und Kontrolle von Forschungs- und Arbeitsergebnissen Dritter, die in die vorgenannten Kategorien fallen.
\end{enumerate}


\section{Vereinstätigkeit}\label{sec:vereinstaetigkeit}

Der Verein erfüllt seine Aufgabe durch
\begin{enumerate}
\item Planung, Durchführung und Protokollierung von Studien
\item Veröffentlichung von Arbeitsergebnissen 
\item Erneute Durchführung und Veröffentlichung Studien Dritter zu deren Verifizierung oder Falsifizierung
\item Kritische Kommentierung von wissenschaftlichen Arbeiten
\item Veröffentlichung begründeter Ablehnung nicht falsifizierbarer Hypothesen und Theorien.
\end{enumerate}


\section{Eintragung in das Vereinsregister}\label{sec:eintragung_in_das_vereinsregister}

Der Verein soll zu einem späteren Zeitpunkt in das Vereinsregister eingetragen werden.


\section{Eintritt der Mitglieder}\label{sec:eintritt_der_mitglieder}

\begin{enumerate}
\item Mitglied des Vereins kann jede voll geschäftsfähige natürliche Person werden.
\item Die Mitgliedschaft entsteht durch Eintritt in den Verein.
\item Ein Mitgliedsantrag ist dem Vorstand schriftlich (\ref{sec:schriftform} Schriftform) vorzulegen.
\item Über die Aufnahme entscheidet der Vorstand. Der Eintritt wird mit Aushändigung einer schriftlichen Aufnahmeerklärung und einem Initiationsritus wirksam. Genaueres regelt die Initiationsritenverordnung. Gibt es keine Initiationsritenverordnung, so entscheidet "`die Willkür"' über das Vorgehen.
\item Ein Aufnahmeanspruch besteht nicht.
\item Der Vorstand kann ohne vorherig eingereichten schriftlichen Antrag beschließen, dass einer bestimmten Person der Beitritt erlaubt werden würde. In einem solchen Fall beginnt die Mitgliedschaft dieser Person durch deren Willensäußerung gegenüber einem Vorstandsmitglied.
\end{enumerate}


\section{Austritt der Mitglieder}\label{sec:austritt_der_mitglieder}

\begin{enumerate}
\item Die Mitglieder sind zum Austritt aus dem Verein berechtigt.
\item Der Austritt erfolgt mit fünfwöchiger Frist zum 23. eines jeden Monats.\label{para:austrittsfrist}
\item Der Austritt ist dem Vorstand schriftlich (\ref{sec:schriftform} Schriftform) zu erklären. Zur Einhaltung der Kündigungsfrist (Absatz \ref{para:austrittsfrist}) ist rechtzeitiger Zugang der Austrittserklärung an ein Mitglied des Vorstands erforderlich.
\item Die Mitgliedschaft ist nicht übertragbar.
\end{enumerate}


\section{Ausschluss der Mitglieder}\label{sec:ausschluss_der_mitglieder}

\begin{enumerate}
\item Die Mitgliedschaft endet außerdem durch Ausschluss.
\item Über den Ausschluss entscheidet auf Antrag des Vorstands die Mitgliederversammlung.
\item Der Vorstand hat seinen Antrag dem auszuschließenden Mitglied mindestens zwei Wochen vor der Versammlung schriftlich (\ref{sec:schriftform} Schriftform) mitzuteilen.
\item Eine schriftlich (\ref{sec:schriftform} Schriftform) eingehende Stellungnahme des Mitglieds ist in der über den Ausschluss entscheidenden Versammlung zu verlesen.
\item Der Ausschluss eines Mitglieds wird sofort mit der Beschlussfassung wirksam.
\item Der Ausschluss soll dem Mitglied, wenn es bei Beschlussfassung nicht anwesend war, durch den Vorstand unverzüglich eingeschrieben bekanntgemacht werden.
\end{enumerate}


\section{Mitgliedsbeitrag}\label{sec:mitgliedsbeitrag}

\begin{enumerate}
\item Der Mitgliedsbeitrag wird durch die Beitragsordnung geregelt.
\item In bedauerlichen Einzelfällen kann mit dem Vorstand eine individuelle Beitragsregelung vereinbart werden.
\end{enumerate}


\section{Organe des Vereins}\label{sec:organe_des_vereins}

Organe des Vereins sind
\begin{enumerate}
\item der Vorstand (\ref{sec:vorstand} und \ref{sec:beschraenkung_der_vertretungsmacht_des_vorstands}),
\item die Mitgliederversammlung (\ref{sec:berufung_der_mitgliederversammlung} bis \ref{sec:beurkundung_der_versammlungsbeschluesse}),
\item die Willkür (\ref{sec:die_willkuer})
\end{enumerate}


\section{Vorstand}\label{sec:vorstand}

\begin{enumerate}
\item Der Vorstand (§ 26 BGB) besteht aus dem 1. Vorsitzenden, dem 2. Vorsitzenden und dem Schriftführer und optional einem Kassier.
\item Je zwei Vorstandsmitglieder vertreten den Verein gemeinsam.
\item Der Vorstand wird durch Beschluss der Mitgliederversammlung bestellt. Er bleibt bis zur satzungsgemäßen Bestellung des nächsten Vorstands im Amt.
\item Das Amt eines Mitglieds des Vorstands endet mit seinem Ausscheiden aus dem Verein.
\item Verschiedene Vorstandsämter können nicht in einer Person vereinigt werden.
\end{enumerate}


\section{Beschränkung der Vertretungsmacht des Vorstands}\label{sec:beschraenkung_der_vertretungsmacht_des_vorstands}

Die Vertretungsmacht des Vorstands ist mit Wirkung gegen Dritte in der Weise beschränkt (§ 26 Abs. 2 Satz 2 BGB), dass zum Erwerb oder Verkauf, zur Belastung und zu allen sonstigen Verfügungen über Grundstücke (und grundstücksgleiche Rechte) sowie außerdem zur Aufnahme eines Kredits von mehr als 1.000,- (m.W.: eintausend) Euro die Zustimmung der Mitgliederversammlung erforderlich ist.


\section{Berufung der Mitgliederversammlung}\label{sec:berufung_der_mitgliederversammlung}

\begin{enumerate}
\item Die Mitgliederversammlung ist zu berufen, wenn es das Interesse des Vereins erfordert, jedoch mindestens\label{para:berufungsvorgang}
    \begin{enumerate}
    \item jährlich einmal, möglichst in den ersten drei Monaten des Kalenderjahres,\label{para:berufungsfrequenz}
    \item nach Ausscheiden eines Mitglieds des Vorstands binnen 3 Monaten.
    \end{enumerate}
\item In einem Jahr, in dem keine Vorstandswahl stattfindet, haben der Vorstand der nach Abs. \ref{para:berufungsvorgang} Buchstabe \ref{para:berufungsfrequenz} zu berufenden Versammlung einen Jahresbericht und eine Jahresabrechnung vorzulegen und die Versammlung über die Entlastung des Vorstands Beschluss zu fassen.
\end{enumerate}


\section{Form der Berufung}\label{sec:form_der_berufung}

\begin{enumerate}
\item Die Mitgliederversammlung ist vom Vorstand schriftlich (\ref{sec:schriftform} Schriftform) unter Einhaltung einer Frist von 2 Wochen zu berufen.
\item Die Berufung der Versammlung muss den Gegenstand der Beschlussfassung (= die Tagesordnung) bezeichnen.
\item Die Frist beginnt mit dem Tag der Absendung der Einladung an die letzte bekannte Mitgliederanschrift (\ref{sec:schriftform} Schriftform).
\end{enumerate}


\section{Beschlussfähigkeit}\label{sec:beschlussfaehigkeit}

\begin{enumerate}
\item Beschlussfähig ist jede ordnungsgemäß berufene Mitgliederversammlung.\label{para:beschlussfaehig_ordnungsgemaess}
\item Zur Beschlussfassung über die Auflösung des Vereins (§ 41 BGB) ist die Anwesenheit (\ref{sec:anwesenheit}~Anwesenheit) von zwei Dritteln der Vereinsmitglieder erforderlich.\label{para:beschlussfaehig_aufloesungsmehrheit}
\item Ist eine zur Beschlussfassung über die Auflösung des Vereins einberufene Mitgliederversammlung nach Absatz \ref{para:beschlussfaehig_aufloesungsmehrheit} nicht beschlussfähig, so ist vor Ablauf von 4 Wochen seit dem Versammlungstag eine weitere Mitgliederversammlung mit derselben Tagesordnung einzuberufen.\label{para:beschlussfaehig_alternative_mv}
\item Die weitere Versammlung darf frühestens 2 Monate nach dem ersten Versammlungstag stattfinden, hat aber jedenfalls spätestens 4 Monate nach diesem Zeitpunkt zu erfolgen.
\item Die Einladung zu der weiteren Versammlung hat einen Hinweis auf die erleichterte Beschlussfähigkeit (Absatz \ref{para:beschluss_neue_versammlung}) zu enthalten.
\item Die neue Versammlung ist ohne Rücksicht auf die Zahl der anwesenden (\ref{sec:anwesenheit}~Anwesenheit) Vereinsmitglieder beschlussfähig.\label{para:beschluss_neue_versammlung}
\end{enumerate}


\section{Beschlussfassung}\label{sec:beschlussfassung}

\begin{enumerate}
\item Es wird durch unmissverständliche Zeichengebung abgestimmt. Auf Antrag ist schriftlich und geheim abzustimmen.\label{para:beschluss_zeichen}
\item Bei der Beschlussfassung entscheidet die Mehrheit der anwesenden (\ref{sec:anwesenheit} Anwesenheit) Mitglieder.\label{para:beschluss_anwesende_mehrheit}
\item Zu einem Beschluss, der eine Änderung der Satzung enthält, ist eine Mehrheit von drei Vierteln der anwesenden (\ref{sec:anwesenheit} Anwesenheit) Mitglieder erforderlich.\label{para:beschluss_satzungsaenderungsmehrheit}
\item Zur Änderung des Zwecks des Vereins (\ref{sec:zweck_des_vereins} der Satzung) ist die Zustimmung aller Mitglieder erforderlich; die Zustimmung der nicht anwesenden Mitglieder muss schriftlich (\ref{sec:schriftform} Schriftform) erfolgen.\label{para:beschluss_zweckaenderungsmehrheit}
\item Zur Beschlussfassung über die Auflösung des Vereins (§ 41 BGB) ist eine Mehrheit von vier Fünfteln der anwesenden (\ref{sec:anwesenheit} Anwesenheit) Mitglieder erforderlich.\label{para:beschluss_aufloesungsmehrheit}
\item Stimmenthaltungen zählen für die Mehrheiten der anwesenden (\ref{sec:anwesenheit} Anwesenheit) Mitglieder (Absätze \ref{para:beschluss_anwesende_mehrheit}, \ref{para:beschluss_satzungsaenderungsmehrheit} und \ref{para:beschluss_aufloesungsmehrheit}) als NEIN-Stimmen.
\end{enumerate}


\section{Beurkundung der Versammlungsbeschlüsse}\label{sec:beurkundung_der_versammlungsbeschluesse}

\begin{enumerate}
\item Über die in der Versammlung gefassten Beschlüsse ist eine Niederschrift aufzunehmen.
\item Die Niederschrift ist von dem Vorsitzenden der Versammlung zu Unterschreiben. Wenn mehrere Vorsitzende tätig waren, unterzeichnet der letzte Versammlungsleiter die ganze Niederschrift.
\item Jedes Vereinsmitglied ist berechtigt, die Niederschrift einzusehen.

\end{enumerate}


\section{Die Willkür}\label{sec:die_willkuer}

\begin{enumerate}
\item Die "`Willkür"' wirkt als Aufsichtsrat des Vereins. Sie besteht aus Mitgliedren des Vereins. Vorstandsmitglieder können ebenfalls Mitglieder der "`Willkür"' sein. Die Willkür wird von der Mitgliederversammlung bestellt.
\item Ihre Aufgaben sind insbesondere:
\begin{enumerate}
\item Beratung des Vorstandes bei strategischen Entscheidungen.
\item Entscheidung über Initiationsriten.
\item Festlegung des Protokolls
\item Benennung von Referaten und Arbeitsgruppen zur Organisation
\end{enumerate}
\end{enumerate}


\section{Schriftform}\label{sec:schriftform}

\begin{enumerate}
\item Die Schriftform ist grundsätzlich durch E-Mail, Jabber und IRC-Absprachen gewahrt.
\item Optional kann von beteiligten Parteien die Authentifizierung aller beteiligten Parteien durch entsprechende digitale Signatur verlangt werden.
\end{enumerate}


\section{Anwesenheit}\label{sec:anwesenheit}

Anwesenheit ist auch durch Telepräsenzmittel gewahrt.


\section{Auflösung des Vereins}\label{sec:aufloesung_des_vereins}

\begin{enumerate}
\item Der Verein kann durch Beschluss der Mitgliederversammlung (vgl. \ref{sec:beschlussfassung} Abs.~\ref{para:beschlussfaehig_aufloesungsmehrheit}) aufgelöst werden.
\item Die Liquidation erfolgt durch den Vorstand (\ref{sec:vorstand}),
\item Das Vereinsvermögen fällt an die Wau Holland Stiftung, Postfach~65~04~43, 22364 Hamburg
\end{enumerate}


\bigskip 

Stuttgart, den 10. Dezember 2012.

\medskip 

Unterschriften der Gründungsmitglieder (genau 3):


\newcommand{\signature}[1]{
\rule[-1.5cm]{11.5cm}{1pt}\\
\hspace*{13pt}({#1})
}


\signature{Bauer, Armin - Referat für feinmechanische Arbeiten an Überbleibseln der 70er und 80er Jahre, der kürzeren Vergangenheit und nahen und fernen Zukunft, Urbanstr. 77, 70190 Stuttgart}

\signature{Jehle, Gregor - kombinat 23, Heinrich-Baumann-Straße 15, 70190 Stuttgart}

\signature{Pommerenke, Lasse - kombinat 23, Heinrich-Baumann-Straße 15, 70190 Stuttgart}



\end{document}
